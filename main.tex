\documentclass{article}
\usepackage[utf8]{inputenc}

\begin{document}

\title{Query-based Paradigms for HEP Analysis}

\maketitle

\begin{abstract}
HEP analysis is typically procedural: users implement arbitrary code in a procedural language, typically oriented toward a piece of code that is executed per-event.  We believe the HEP community investigate in an analysis ecosystem that is oriented to \textit{query-based paradigms} instead, where users present the computing system with a high-level query; the computing system is then responsible for query compilation, planning, execution, and returning results.  We give example of one such approach, FemtoCode, and discuss how it could be evolved into a larger ecosystem that keeps and leverages investments in today's technologies such as ROOT and distributed computing.
\end{abstract}

\section{Introduction}

``Faster is better."  The adage is a bedrock belief among HEP analysts.  We believe significantly better science can be done if a computing system could return results over a lunch break versus overnight; or over a coffee break as opposed to a lunch break.

One outcome of the ``faster is better'' approach is significant investment in low-level, fast languages: given the timeline of the community's involvement in large-scale research computing, this has implied we have an ecosystem rooted in C++.  Along with other scientific communities, there has a been more recent uptake in Python as a scripting language for exploring datasets.

Straight translation of C++ code to Python is likely to lead to a miserable user experience: the resulting program, while quicker to write for a typical graduate student, will be 10-100x slower.  This is because python is both a general purpose and a high-level language.  We propose the HEP community investigate in a domain-specific languages that have a narrow-purpose but expose high-level abstractions needed by analysts \footnotemark.  We claim:

\footnotetext{This approach is not foreign: GPU implementation of code is often significantly faster than CPU implementations because GPUs provide a more limited programming environment amenable to acceleration.}

\begin{itemize}
\item Such a domain-specific language (DSL) can be embedded inside a more well known language like python.
\item The DSL can be accelerated to be faster than C++/ROOT/event-loop techniques used today.
\item The overarching approach can leverage today's file formats and distributed computing infrastructure, but providing the highest possible event rates may require specialized computing resources.
\end{itemize}

To justify, this claim, we present preliminary results from the DIANA/HEP team's work on FemtoCode and HistoGrammar, then discuss how this can evolve from existing computing models.

\section{FemtoCode and HistoGrammar: A short overview}

TODO: give the most high-level overview of the two concepts.  MAX ONE PAGE.  Should involve a simple example that touches both.

\section{Preliminary FemtoCode Results}

TODO: Pick the favorite result from the poster and copy/paste it here.  Result should have impact and be straightforward to explain.

\section{Computing Model Discussion}

TODO: Approaches like FemtoCode / HistoGrammar don't necessarily imply we need to dump ROOT and those corresponding investments, but we can continuously evolve on top of that base.

\end{document}
